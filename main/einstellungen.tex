%%%%%%%%%%%%%%%%%%%%%%%%%%%%%%%%%%%%%%%%%%%%%%%%%%%%%%%%%%%%%%%%%%%%%%%%%%%%%%%
%                                   Einstellungen
% Hier können alle relevanten Einstellungen für diese Arbeit gesetzt werden.
% Dazu gehören Angaben u.a. über den Autor sowie Formatierungen.
%%%%%%%%%%%%%%%%%%%%%%%%%%%%%%%%%%%%%%%%%%%%%%%%%%%%%%%%%%%%%%%%%%%%%%%%%%%%%%%

%%%%%%%%%%%%%%%%%%%%%%%%%%%%%%%%%%%% Sprache %%%%%%%%%%%%%%%%%%%%%%%%%%%%%%%%%%%
%% Aktuell sind Deutsch und Englisch unterstützt.
%% Es werden nicht nur alle vom Dokument erzeugten Texte in
%% der entsprechenden Sprache angezeigt, sondern auch weitere
%% Aspekte angepasst, wie z.B. die Anführungszeichen und
%% Datumsformate.
\setzesprache{de} % de oder en
%%%%%%%%%%%%%%%%%%%%%%%%%%%%%%%%%%%%%%%%%%%%%%%%%%%%%%%%%%%%%%%%%%%%%%%%%%%%%%%%

%%%%%%%%%%%%%%%%%%%%%%%%%%%%%%%%%%% Angaben  %%%%%%%%%%%%%%%%%%%%%%%%%%%%%%%%%%%
%% Die meisten der folgenden Daten werden auf dem Deckblatt angezeigt, einige 
%% auch im weiteren Verlauf des Dokuments.
\setzematrikelnr{-todo-}
\setzekurs{-todo-}      % Kursbezeichnung an der DHBW
\setzetitel{-todo-}     % Titel der (Projekt-)Arbeit
\setzedatumAnfang{-todo-}
\setzedatumAbgabe{-todo-}
\setzefirma{-todo-}
\setzefirmenort{-todo-}
\setzeabgabeort{-todo-} % für Unterschriftenfeld
\setzephase{\"uber die Praxisphasen des ersten Studienjahres} % ersten, zweiten, dritten Studienjahres, bei BA leer lassen
\setzestudiengang{-todo-}
\setzedhbw{-todo-}
\setzecampus{-todo-}
\setzebetreuer{-todo-}
\setzegutachter{-todo-}
\setzezeitraum{-todo-}
\setzearbeit{-todo-}    % bspw. Projektarbeit T1000
\setzeautor{-todo-}
\setzesemester{-todo-}
\setzeschwerpunkt{-todo-}
\setzejahrgang{-todo-}
\setzeabteilung{-todo-}
\setzestandort{-todo-}

% jeweilige Zeilen auskommentieren oder zu \....false ändern, falls entsprechendes Objekt nicht benötigt wird
\inhalttrue
\selbsterkltrue
\sperrvermerktrue
\abkverztrue                
\abbverztrue               
\tableverztrue            
\listverztrue              
\formelverztrue 
\abstracttrue          
\bothabstractsfalse        
\appendixtrue               
\literaturtrue              % auskommentieren oder ändern zu \literaturfalse, wenn kein Literaturverzeichnis gewünscht ist (\appendixtrue muss gesetzt sein!)
\refWithPagesfalse          % ändern zu \refWithPagestrue, wenn die Seitenzahl bei Verweisen auf Kapitel eingefügt werden sollen
\reviewertrue				% Zeile für Gutachter auf Deckblatt

% Angabe des roten/gelben Punktes auf dem Titelblatt zur Kennzeichnung der Vertraulichkeitsstufe.
% Mögliche Angaben sind \yellowdottrue und \reddottrue. Werden beide angegeben, wird der rote Punkt gezeichnet.
% Wird keines der Kommandos angegeben erscheint ein grüner Punkt (s. Deckblatt.tex)
%\yellowdottrue

%%%%%%%%%%%%%%%%%%%%%%%%%%%%%%%%%%%%%%%%%%%%%%%%%%%%%%%%%%%%%%%%%%%%%%%%%%%%%%%%
%%%%%%%%%%%%%%%%%%%%%%%%%%%% Literaturverzeichnis %%%%%%%%%%%%%%%%%%%%%%%%%%%%%%
%% Bei Fehlern während der Verarbeitung bitte in ads/header.tex bei der
%% Einbindung des Pakets biblatex (ungefähr ab Zeile 110,
%% einmal für jede Sprache), biber in bibtex ändern.
\newcommand{\ladeliteratur}{%
\addbibresource{bibliographie.bib}
%\addbibresource{weitereDatei.bib}
}

%% Zitierstil
%% siehe: http://ctan.mirrorcatalogs.com/macros/latex/contrib/biblatex/doc/biblatex.pdf (3.3.1 Citation Styles)
%% mögliche Werte z.B numeric-comp, alphabetic, authoryear
\setzezitierstil{alphabetic}

%%%%%%%%%%%%%%%%%%%%%%%%%%%%%%%%% Layout %%%%%%%%%%%%%%%%%%%%%%%%%%%%%%%%%%%%%%%
%% Verschiedene Schriftarten
% laut nag Warnung: palatino obsolete, use mathpazo, helvet (option scaled=.95), courier instead
\setzeschriftart{lmodern} % palatino oder goudysans, lmodern, libertine

%% Abstand vor Kapitelüberschriften zum oberen Seitenrand
\setzekapitelabstand{20pt}

%% Spaltenabstand
\setzespaltenabstand{10pt}

%% Zeilenabstand innerhalb einer Tabelle
\setzezeilenabstand{1.5}
%%%%%%%%%%%%%%%%%%%%%%%%%%%%%%%%%%%%%%%%%%%%%%%%%%%%%%%%%%%%%%%%%%%%%%%%%%%%%%%%

%%%%%%%%%%%%%%%%%%%%%%%%%%%%%%%%%%%%%%%%%%%%%%%%%%%%%%%%%%%%%%%%%%%%%%%%%%%%%%%%
%%%%%%%%%%%%%%%%%%%%%%%%%%%% eigene Kommandos %%%%%%%%%%%%%%%%%%%%%%%%%%%%%%%%%%
% \newcommand{\cmd}{{\Dieser Text erscheint bei Eingabe von \cmd im Text.}}}