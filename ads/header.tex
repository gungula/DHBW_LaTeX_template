%!TEX root = ../dokumentation.tex

\RequirePackage[l2tabu, orthodox]{nag}	% weist in Commandozeile bzw. log auf veraltete LaTeX Syntax hin

% Einstellungen laden
\usepackage{xstring}
\usepackage{ifpdf}
\usepackage{ifluatex}
\usepackage{lastpage}
\usepackage{fancyhdr}
\newcommand{\einstellung}[1]{%
    \expandafter\newcommand\csname #1\endcsname{}
    \expandafter\newcommand\csname setze#1\endcsname[1]{\expandafter\renewcommand\csname#1\endcsname{##1}}
}
\newcommand{\langstr}[1]{\einstellung{lang#1}}

% Flag für die Selbstständigkeitserklärung, Default: true
\newif\ifselbsterkl
\selbsterklfalse

% Flag für roten Vertraulichkeitspunkt, default: false
\newif\ifreddot
\reddotfalse

% Flag für gelben Vertraulichkeitspunkt, default: false
\newif\ifyellowdot
\yellowdotfalse

% Flag für das Unterschriftenblatt, default: false
\newif\ifunterschriftenblatt
\unterschriftenblattfalse

% Flag für Einfügen der Seitenzahl bei Verweis auf Kapitel/Abschnitt, default: false
\newif\ifrefWithPages
\refWithPagesfalse

% Flag für Einfügen der Abstracts in deutsch und englisch, default: false
\newif\ifbothabstracts
\bothabstractsfalse

% Flag für Einfügen des Abkürzungsverzeichnis
\newif\ifabkverz
\abkverzfalse

% Flag für Einfügen des Abbildungsverzeichnisses
\newif\ifabbverz
\abbverzfalse

% Flag für Einfügen des Formelverzeichnisses
\newif\ifformelverz
\formelverzfalse

% Flag für Einfügen des Listingverzeichnisses
\newif\iflistverz
\listverzfalse

% Flag für Einfügen des Tabellenverzeichnisses
\newif\iftableverz
\tableverzfalse

% Flag für Einfügen des Sperrvermerks
\newif\ifsperrvermerk
\sperrvermerkfalse

% Flag für Einfügen des Abstracts
\newif\ifabstract
\abstractfalse

% Flag für Anhang
\newif\ifappendix
\appendixfalse

% Flag für Literaturverzeichnis
\newif\ifliteratur
\literaturfalse

% Flag für Inhaltsverzeichnis
\newif\ifinhalt
\inhaltfalse

% Flag für Reviewer
\newif\ifreviewer
\reviewerfalse

\einstellung{matrikelnr}
\einstellung{titel}
\einstellung{kurs}
\einstellung{datumAnfang}
\einstellung{datumAbgabe}
\einstellung{firma}
\einstellung{firmenort}
\einstellung{abgabeort}
\einstellung{phase}
\einstellung{studiengang}
\einstellung{dhbw}
\einstellung{campus}
\einstellung{betreuer}
\einstellung{gutachter}
\einstellung{zeitraum}
\einstellung{arbeit}
\einstellung{autor}
\einstellung{sprache}
\einstellung{schriftart}
\einstellung{kapitelabstand}
\einstellung{spaltenabstand}
\einstellung{zeilenabstand}
\einstellung{zitierstil}
\einstellung{selbsterkl}
\einstellung{semester}
\einstellung{schwerpunkt}
\einstellung{jahrgang}
\einstellung{abteilung}
\einstellung{standort}
 % verfügbare Einstellungen
%%%%%%%%%%%%%%%%%%%%%%%%%%%%%%%%%%%%%%%%%%%%%%%%%%%%%%%%%%%%%%%%%%%%%%%%%%%%%%%
%                                   Einstellungen
% Hier können alle relevanten Einstellungen für diese Arbeit gesetzt werden.
% Dazu gehören Angaben u.a. über den Autor sowie Formatierungen.
%%%%%%%%%%%%%%%%%%%%%%%%%%%%%%%%%%%%%%%%%%%%%%%%%%%%%%%%%%%%%%%%%%%%%%%%%%%%%%%

%%%%%%%%%%%%%%%%%%%%%%%%%%%%%%%%%%%% Sprache %%%%%%%%%%%%%%%%%%%%%%%%%%%%%%%%%%%
%% Aktuell sind Deutsch und Englisch unterstützt.
%% Es werden nicht nur alle vom Dokument erzeugten Texte in
%% der entsprechenden Sprache angezeigt, sondern auch weitere
%% Aspekte angepasst, wie z.B. die Anführungszeichen und
%% Datumsformate.
\setzesprache{de} % de oder en
%%%%%%%%%%%%%%%%%%%%%%%%%%%%%%%%%%%%%%%%%%%%%%%%%%%%%%%%%%%%%%%%%%%%%%%%%%%%%%%%

%%%%%%%%%%%%%%%%%%%%%%%%%%%%%%%%%%% Angaben  %%%%%%%%%%%%%%%%%%%%%%%%%%%%%%%%%%%
%% Die meisten der folgenden Daten werden auf dem Deckblatt angezeigt, einige 
%% auch im weiteren Verlauf des Dokuments.
\setzematrikelnr{-todo-}
\setzekurs{-todo-}      % Kursbezeichnung an der DHBW
\setzetitel{-todo-}     % Titel der (Projekt-)Arbeit
\setzedatumAnfang{-todo-}
\setzedatumAbgabe{-todo-}
\setzefirma{-todo-}
\setzefirmenort{-todo-}
\setzeabgabeort{-todo-} % für Unterschriftenfeld
\setzephase{\"uber die Praxisphasen des ersten Studienjahres} % ersten, zweiten, dritten Studienjahres, bei BA leer lassen
\setzestudiengang{-todo-}
\setzedhbw{-todo-}
\setzecampus{-todo-}
\setzebetreuer{-todo-}
\setzegutachter{-todo-}
\setzezeitraum{-todo-}
\setzearbeit{-todo-}    % bspw. Projektarbeit T1000
\setzeautor{-todo-}
\setzesemester{-todo-}
\setzeschwerpunkt{-todo-}
\setzejahrgang{-todo-}
\setzeabteilung{-todo-}
\setzestandort{-todo-}

% jeweilige Zeilen auskommentieren oder zu \....false ändern, falls entsprechendes Objekt nicht benötigt wird
\inhalttrue
\selbsterkltrue
\sperrvermerktrue
\abkverztrue                
\abbverztrue               
\tableverztrue            
\listverztrue              
\formelverztrue 
\abstracttrue          
\bothabstractsfalse        
\appendixtrue               
\literaturtrue              % auskommentieren oder ändern zu \literaturfalse, wenn kein Literaturverzeichnis gewünscht ist (\appendixtrue muss gesetzt sein!)
\refWithPagesfalse          % ändern zu \refWithPagestrue, wenn die Seitenzahl bei Verweisen auf Kapitel eingefügt werden sollen
\reviewertrue				% Zeile für Gutachter auf Deckblatt

% Angabe des roten/gelben Punktes auf dem Titelblatt zur Kennzeichnung der Vertraulichkeitsstufe.
% Mögliche Angaben sind \yellowdottrue und \reddottrue. Werden beide angegeben, wird der rote Punkt gezeichnet.
% Wird keines der Kommandos angegeben erscheint ein grüner Punkt (s. Deckblatt.tex)
%\yellowdottrue

%%%%%%%%%%%%%%%%%%%%%%%%%%%%%%%%%%%%%%%%%%%%%%%%%%%%%%%%%%%%%%%%%%%%%%%%%%%%%%%%
%%%%%%%%%%%%%%%%%%%%%%%%%%%% Literaturverzeichnis %%%%%%%%%%%%%%%%%%%%%%%%%%%%%%
%% Bei Fehlern während der Verarbeitung bitte in ads/header.tex bei der
%% Einbindung des Pakets biblatex (ungefähr ab Zeile 110,
%% einmal für jede Sprache), biber in bibtex ändern.
\newcommand{\ladeliteratur}{%
\addbibresource{bibliographie.bib}
%\addbibresource{weitereDatei.bib}
}

%% Zitierstil
%% siehe: http://ctan.mirrorcatalogs.com/macros/latex/contrib/biblatex/doc/biblatex.pdf (3.3.1 Citation Styles)
%% mögliche Werte z.B numeric-comp, alphabetic, authoryear
\setzezitierstil{alphabetic}

%%%%%%%%%%%%%%%%%%%%%%%%%%%%%%%%% Layout %%%%%%%%%%%%%%%%%%%%%%%%%%%%%%%%%%%%%%%
%% Verschiedene Schriftarten
% laut nag Warnung: palatino obsolete, use mathpazo, helvet (option scaled=.95), courier instead
\setzeschriftart{lmodern} % palatino oder goudysans, lmodern, libertine

%% Abstand vor Kapitelüberschriften zum oberen Seitenrand
\setzekapitelabstand{20pt}

%% Spaltenabstand
\setzespaltenabstand{10pt}

%% Zeilenabstand innerhalb einer Tabelle
\setzezeilenabstand{1.5}
%%%%%%%%%%%%%%%%%%%%%%%%%%%%%%%%%%%%%%%%%%%%%%%%%%%%%%%%%%%%%%%%%%%%%%%%%%%%%%%%

%%%%%%%%%%%%%%%%%%%%%%%%%%%%%%%%%%%%%%%%%%%%%%%%%%%%%%%%%%%%%%%%%%%%%%%%%%%%%%%%
%%%%%%%%%%%%%%%%%%%%%%%%%%%% eigene Kommandos %%%%%%%%%%%%%%%%%%%%%%%%%%%%%%%%%%
% \newcommand{\cmd}{{\Dieser Text erscheint bei Eingabe von \cmd im Text.}}}      % lese Einstellungen
\newcommand{\iflang}[2]{%
  \IfStrEq{\sprache}{#1}{#2}{}
}

\langstr{abkverz}
\langstr{anhang}
\langstr{glossar}
\langstr{deckblattabschlusshinleitung}
\langstr{artikelstudiengang}
\langstr{studiengang}
\langstr{anderdh}
\langstr{campus}
\langstr{von}
\langstr{dbbearbeitungszeit}
\langstr{dbmatriknr}
\langstr{dbkurs}
\langstr{dbfirma}
\langstr{dbbetreuer}
\langstr{dbgutachter}
\langstr{sperrvermerk}
\langstr{abstract}
\langstr{listingname}
\langstr{listlistingname}
\langstr{listingautorefname}
\langstr{selbsterkl}
\langstr{formelsammlung}
\langstr{kopfz}
\langstr{fussz}
\langstr{seite}
\langstr{seitevon}
\langstr{stand}
\langstr{formelgroeverz}            % verfügbare Strings
\input{lang/\sprache}           % Übersetzung einlesen

%\lstset{language=Matlab}
\usepackage{pdfpages}           % pdf-Seiten einbinden


%% Farben (Angabe in HTML-Notation mit großen Buchstaben)
\newcommand{\ladefarben}{%
	\definecolor{LinkColor}{HTML}{00007A}
	\definecolor{ListingBackground}{HTML}{E6E6E6}
}

%% Mathematikpakete benutzen (Pakete aktivieren)
%\usepackage{amsmath}
%\usepackage{amssymb}

%% Programmiersprachen Highlighting (Listings)
\newcommand{\listingsettings}{%
	\lstset{%
		language=C++,			% Standardsprache des Quellcodes
		%numbers=left,			% Zeilennummern links
		%stepnumber=1,			% Jede Zeile nummerieren.
		%numbersep=5pt,			% 5pt Abstand zum Quellcode
		%numberstyle=\tiny,		% Zeichengrösse 'tiny' für die Nummern.
		breaklines=true,		% Zeilen umbrechen wenn notwendig.
		breakautoindent=true,	% Nach dem Zeilenumbruch Zeile einrücken.
		postbreak=\space,		% Bei Leerzeichen umbrechen.
		tabsize=2,				% Tabulatorgrösse 2
		basicstyle=\ttfamily\footnotesize, % Nichtproportionale Schrift, klein für den Quellcode
		showspaces=false,		% Leerzeichen nicht anzeigen.
		showstringspaces=false,	% Leerzeichen auch in Strings ('') nicht anzeigen.
		extendedchars=true,		% Alle Zeichen vom Latin1 Zeichensatz anzeigen.
		captionpos=b,			% sets the caption-position to bottom
		backgroundcolor=\color{ListingBackground}, % Hintergrundfarbe des Quellcodes setzen.
		xleftmargin=0pt,		% Rand links
		xrightmargin=0pt,		% Rand rechts
		frame=single,			% Rahmen an
		frameround=ffff,
		rulecolor=\color{darkgray},	% Rahmenfarbe
		%fillcolor=\color{ListingBackground},
		keywordstyle=\color[rgb]{0.133,0.133,0.6},
		commentstyle=\color[rgb]{0.133,0.545,0.133},
		stringstyle=\color[rgb]{0.627,0.126,0.941},
        aboveskip=1.5em,
	}
}

%%%%%%%%%%%%%%%%%%%%%%%%%%%%% Kopf-/Fußzeilenwechsel %%%%%%%%%%%%%%%%%%%%%%%%%%%
%% Kopf- und Fußzeilen von Verzeichnissen
\setlength{\headheight}{40pt}
\newcommand{\setpagestylehead}{
    \fancypagestyle{plain}{%
        \fancyhf{}
        \fancyhead[L]{\vspace{0.5cm}\small \langkopfz}
        \fancyhead[R]{
            \hspace{2.0cm}
            \iflang{de}{
				\begin{textblock*}{188mm}(-3mm,9mm) %(-50mm,9mm) sind die korrekten Einstellungen, falls zusätzlich ein Firmenlogo in die Kopfzeile eingefügt wird         	
            	\includegraphics[height=1.4cm]{images/dhbw_de}
            	\end{textblock*}
            }
            \iflang{en}{
            	\begin{textblock*}{188mm}(-4mm,8mm) %(-34mm,7mm) sind die korrekten Einstellungen, falls zusätzlich ein Firmenlogo in die Kopfzeile eingefügt wird        	
            	\includegraphics[height=1.2cm]{images/dhbw_en}
            	\end{textblock*}
            }
            %\iflang{de}{
		%		\begin{textblock*}{188mm}(0mm,0mm)            	
            	%\includegraphics[height=2.7cm]{-todo-}  % Logo der Partnerfirma in Kopfzeile von Verzeichnissen
            %	\end{textblock*}
            %}
            %\iflang{en}{
            %	\begin{textblock*}{188mm}(0mm,0mm)            	
            	%\includegraphics[height=2.2cm]{-todo-}  % Logo der Partnerfirma in Kopfzeile von Verzeichnissen
            %	\end{textblock*}
            %}
        }
        \fancyfoot[L]{
            \noindent{\tiny \langfussz\\
                \begin{tabular*}{16cm}{@{\extracolsep{\fill}}l>{\raggedleft}p{8cm}}
                    {\tiny \langstand: \today} & 
                    {\tiny \langseite\ \thepage\ \langseitevon\ \pageref*{endOfRomanNumbering}\vspace{1cm}}\tabularnewline
                \end{tabular*}
            }
        }
    }
    \pagestyle{plain}
    \pagenumbering{roman}
}

%% Kopf- und Fußzeilen von content-Seiten
\newcommand{\setpagestylecontent}{
\fancypagestyle{plain}{%
        \fancyhf{}
        \fancyhead[L]{\vspace{0.5cm}\small \langkopfz}
        \fancyhead[R]{
            \hspace{2.0cm}
            \iflang{de}{
				\begin{textblock*}{188mm}(-3mm,9mm) %(-50mm,9mm) sind die korrekten Einstellungen, falls zusätzlich ein Firmenlogo in die Kopfzeile eingefügt wird         	
            	\includegraphics[height=1.4cm]{images/dhbw_de}
            	\end{textblock*}
            }
            \iflang{en}{
            	\begin{textblock*}{188mm}(-4mm,8mm) %(-34mm,7mm) sind die korrekten Einstellungen, falls zusätzlich ein Firmenlogo in die Kopfzeile eingefügt wird        	
            	\includegraphics[height=1.2cm]{images/dhbw_en}
            	\end{textblock*}
            }
            %\iflang{de}{
		%		\begin{textblock*}{188mm}(0mm,0mm)            	
            	%\includegraphics[height=2.7cm]{-todo-}  % Logo der Partnerfirma in Kopfzeile von Verzeichnissen
            %	\end{textblock*}
            %}
            %\iflang{en}{
            %	\begin{textblock*}{188mm}(0mm,0mm)            	
            	%\includegraphics[height=2.2cm]{-todo-}  % Logo der Partnerfirma in Kopfzeile von Verzeichnissen
            %	\end{textblock*}
            %}
        }
        \fancyfoot[L]{
            \noindent{\tiny \langfussz\\
                \begin{tabular*}{16cm}{@{\extracolsep{\fill}}l>{\raggedleft}p{8cm}}
                    {\tiny \langstand: \today} & 
                    {\tiny \langseite\ \thepage\ \langseitevon\ \pageref*{endOfArabicNumbering}\vspace{1cm}}\tabularnewline
                \end{tabular*}
            }
        }
    }
    \pagestyle{plain}
    \pagenumbering{arabic}
}

%% Kopf- und Fußzeilen von Anhängen 
\newcommand{\setpagestylefoot}{
\fancypagestyle{plain}{%
        \fancyhf{}
        \fancyhead[L]{\vspace{0.5cm}\small \langkopfz}
        \fancyhead[R]{
            \hspace{2.0cm}
            \iflang{de}{
				\begin{textblock*}{188mm}(-3mm,9mm)            	
            	\includegraphics[height=1.4cm]{images/dhbw_de}
            	\end{textblock*}
            }
            \iflang{en}{
            	\begin{textblock*}{188mm}(-4mm,8mm)            	
            	\includegraphics[height=1.2cm]{images/dhbw_en}
            	\end{textblock*}
            }
            \iflang{de}{
				\begin{textblock*}{188mm}(0mm,0mm)            	
                %\includegraphics[height=2.7cm]{-todo-}     % Logo der Partnerfirma in Kopfzeile von Anhängen
            	\end{textblock*}
            }
            \iflang{en}{
            	\begin{textblock*}{188mm}(0mm,0mm)            	
            	%\includegraphics[height=2.2cm]{-todo-}     % Logo der Partnerfirma in Kopfzeile von Anhängen
            	\end{textblock*}
            }
        }
        \fancyfoot[L]{
            \noindent{\tiny \langfussz\\
                \begin{tabular*}{16cm}{@{\extracolsep{\fill}}l>{\raggedleft}p{8cm}}
                    {\tiny \langstand: \today} & 
                    {\tiny \langseite\ \thepage\ \langseitevon\ \pageref*{LastPage}\vspace{1cm}}\tabularnewline
                \end{tabular*}
            }
        }
    }
    \pagestyle{plain}
    \pagenumbering{Alph}
    \renewcommand{\thepage}{\AlphAlph{\value{page}}}
}


%%%%%%%%%%%%%%%%%%%%%%%%%%%%%%%%%%%%%%%%%%%%%%%%%%%%%%%%%%%%%%%%%%%%%%%%%%%%%%%%
%% Einstellung der Sprache des Paketes Babel und der Verzeichnisüberschriften

\iflang{de}{
    \usepackage[english, ngerman]{babel}
    \selectlanguage{ngerman}
}
\iflang{en}{
    \usepackage[ngerman, english]{babel}
    \selectlanguage{english}
}

\usepackage[utf8]{inputenc}
\usepackage[T1]{fontenc}
\usepackage{tikz}
\usepackage[margin=2.5cm,foot=1cm,top=3cm,bottom=3cm]{geometry}	% Seitenränder und Abstände
\usepackage[activate]{microtype} %Zeilenumbruch und mehr
\usepackage[onehalfspacing]{setspace}
\usepackage{makeidx}
\usepackage[autostyle=true,german=quotes]{csquotes}
\usepackage{longtable}
\usepackage{enumitem}	% mehr Optionen bei Aufzählungen
\usepackage{graphicx}
\usepackage{xcolor} 	% für HTML-Notation
\usepackage{float}
\usepackage{array}
\usepackage{calc}		% zum Rechnen (Bildtabelle in Deckblatt)
\usepackage[right]{eurosym}
\usepackage{wrapfig}
\usepackage{pgffor}     % für automatische Kapiteldateieinbindung
\usepackage[perpage, hang, multiple, stable]{footmisc} % Fussnoten
\usepackage{acronym}
\usepackage[absolute]{textpos}
%\usepackage[printonlyused, footnote]{acronym} % falls gewünscht kann die Option footnote eingefügt werden, dann wird die Erklärung nicht inline sondern in einer Fußnote dargestellt
\usepackage{scrhack} % in Kombination mit listings-Package kommt es zu Warnings, dieses Paket verhindert die Warnings! Ggf. auskommentieren und die Warnings akzeptieren falls Verzeichnisse nicht so dargestellt werden wie gewünscht
\usepackage{listings} % Code-Listings
%\usepackage[numbered, framed]{matlab-prettifier}
\usepackage[framed]{matlab-prettifier} % .sty-Datei muss vorhanden sein! Kann auskommentiert werden, falls keine Matlab-Listings in der Arbeit enthalten sind.
\usepackage{color, colortbl}  % Für Highlighten der Tabellenzeilen
\usepackage{amsmath}% http://ctan.org/pkg/amsmath
\usepackage{alphalph}

% eine Kommentarumgebung "k" (Handhabe mit \begin{k}<Kommentartext>\end{k},
% Kommentare werden rot gedruckt). Wird \% vor excludecomment{k} entfernt,
% werden keine Kommentare mehr gedruckt.
\usepackage{comment}
\specialcomment{k}{\begingroup\color{red}}{\endgroup}
%\excludecomment{k}


%%%%%% Konfiguration %%%%%%
%% Anwenden der Einstellungen
\usepackage{\schriftart}
\ladefarben{}

% Titel, Autor und Datum
\title{\titel}
\author{\autor}
\date{\datum}

%\usepackage[list=true]{subcaption}

% PDF Einstellungen
\usepackage[%
    pdftitle={\titel},
    pdfauthor={\autor},
    pdfsubject={\arbeit},
    pdfcreator={pdflatex, LaTeX with KOMA-Script},
    pdfpagemode=UseOutlines, 		% Beim Öffnen Inhaltsverzeichnis anzeigen
    pdfdisplaydoctitle=true, 		% Dokumententitel statt Dateiname anzeigen.
    pdflang={\sprache}, 			% Sprache des Dokuments.
]{hyperref}

% (Farb-)Einstellungen für die Links im PDF
\hypersetup{%
    colorlinks=true, 		% Aktivieren von farbigen Links im Dokument
    linkcolor=black, 	    % Farbe festlegen
    citecolor=LinkColor,
    filecolor=LinkColor,
    menucolor=LinkColor,
    urlcolor=LinkColor,
    %linktocpage=true, 		% Nicht der Text sondern die Seitenzahlen in Verzeichnissen klickbar
    linktoc=all,            % Seitenzahlen und Text klickbar
    bookmarksnumbered=true 	% Überschriftsnummerierung im PDF Inhalt anzeigen.
}
% Workaround um Fehler in Hyperref, muss hier stehen bleiben
\usepackage{bookmark} % nur ein latex-Durchlauf für die Aktualisierung von Verzeichnissen nötig

% Schriftart in Captions etwas kleiner
\addtokomafont{caption}{\small}
% Caption nur als "Abb.", nicht als "Abbildung"
\iflang{de}{
    \usepackage[
        figurename=Abb.
    ]{caption}
}
\iflang{en}{
    \usepackage[
        figurename=Fig.
    ]{caption}
}

\usepackage{subfig}

% Literaturverweise (sowohl deutsch als auch englisch)
\iflang{de}{%
\usepackage[
    backend=bibtex,		% empfohlen. Falls biber Probleme macht: bibtex
    bibwarn=true,
    bibencoding=utf8,	% wenn .bib in utf8, sonst ascii
    sortlocale=de_DE,
    style=\zitierstil,
    sorting=none,
]{biblatex}
}
\iflang{en}{%
\usepackage[
    backend=bibtex,		% empfohlen. Falls biber Probleme macht: bibtex
    bibwarn=true,
    bibencoding=utf8,	% wenn .bib in utf8, sonst ascii
    sortlocale=en_US,
    style=\zitierstil,
    sorting=none,
]{biblatex}
}


\ladeliteratur{}
%\bibliography{bibliographie}

% Glossar
\usepackage[nonumberlist,toc]{glossaries}
\usepackage{blindtext} % Blindtext-Package. Common Usage: \blindtext für einzelnen Abschnitt, \Blindtext für mehrere Abschnitte


%%%%%% Additional settings %%%%%%
% Hurenkinder und Schusterjungen verhindern
\clubpenalty = 10000 % schließt Schusterjungen aus (Seitenumbruch nach der ersten Zeile eines neuen Absatzes)
\widowpenalty = 10000 % schließt Hurenkinder aus (die letzte Zeile eines Absatzes steht auf einer neuen Seite)
\displaywidowpenalty=10000

\setcounter{biburlnumpenalty}{100}
\setcounter{biburlucpenalty}{100}
\setcounter{biburllcpenalty}{100}

% Bildpfad
\graphicspath{{images/}}

% Einige häufig verwendete Sprachen
\lstloadlanguages{PHP,Python,Java,C,C++,bash}
\listingsettings{}
% Umbennung des Listings
\renewcommand\lstlistingname{\langlistingname}
\renewcommand\lstlistlistingname{\langlistlistingname}
\def\lstlistingautorefname{\langlistingautorefname}

% Abstände in Tabellen
\setlength{\tabcolsep}{\spaltenabstand}
\renewcommand{\arraystretch}{\zeilenabstand}

\usepackage{xspace}
\newcommand{\lastcontentpage}{}
\usepackage{amsfonts}

%\usetikzlibrary{shapes,arrows,calc}
\usepackage{relsize}

\usepackage{censor}

\usepackage{eso-pic}


%% Paket, um Textteile drehen zu können
%\usepackage{rotating}
%% Paket, um Seite im Querformat anzuzeigen
%\usepackage{lscape}


\ifrefWithPages
    %RJG8FE: add a pageref to autoref whenever the referenced page is not the same as the current one
    %        useful for printed documents without clickable hyperlinks
    \AtBeginDocument{\let\oldautoref\autoref}
    \AtBeginDocument{
        \renewcommand{\autoref}[1]{%
            \oldautoref{#1}%
            \ifthenelse{\thepage=\pageref{#1}}% if current page number equals the referenced page number
            {}% then add nothing
            { (S. \pageref{#1})}% else add the text
        }
    }
\fi

\usepackage{amssymb} % Erweiterung der Symbole in Mathematikumgebung

\iflang{de}{\usepackage{icomma}} % Europäisches Komma in Formeln
